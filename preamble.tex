%%%%%%%%%%%%%%%%%%%%%%%%%%%%%%%%%%%%%%
% Giorgio Caizzi, 2021
%%%%%%%%%%%%%%%%%%%%%%%%%%%%%%%%%%%%%%

% ----------------------
% DOCUMENT CONFIGURATION
% ----------------------

%  font size: 11pt
\documentclass[letterpaper,11pt]{article}

% latexsym package included for \LaTeX logo
\usepackage{latexsym}
\usepackage[empty]{fullpage}
\usepackage{titlesec}
\usepackage{marvosym}
\usepackage[usenames,dvipsnames]{color}
\usepackage{verbatim}
\usepackage{enumitem}
\usepackage[hidelinks]{hyperref}
\usepackage{fancyhdr}
\usepackage[english]{babel}
\usepackage{tabularx}
\usepackage{xifthen}
\usepackage{fontawesome}


\pagestyle{fancy}
\fancyhf{} % clear all header and footer fields
\fancyfoot{}
\renewcommand{\headrulewidth}{0pt}
\renewcommand{\footrulewidth}{0pt}

% Adjust margins
\addtolength{\oddsidemargin}{-0.5in}
\addtolength{\evensidemargin}{-0.5in}
\addtolength{\textwidth}{1in}
\addtolength{\topmargin}{-.5in}
\addtolength{\textheight}{1.0in}

\urlstyle{same}

\raggedbottom
\raggedright
\setlength{\tabcolsep}{0in}

% Sections formatting
\titleformat{\section}{
  \vspace{-10pt}\scshape\raggedright\large
}{}{0em}{}[\color{black}\titlerule \vspace{-5pt}]


% -------------------------
% CUSTOM COMMANDS
% -------------------------

% The CV is made of a header section with info and sections
% each section is made of a list of subheadings 
% that are a table with basic information,
% each subHeading can have a list of items consising 
% in a bullet point list of text.

% -------------------------
% subHeadingList

\newcommand{\subHeadingListStart}{\begin{itemize}[leftmargin=0.15in, label={}]}
\newcommand{\subHeadingListEnd}{\end{itemize}}

% -------------------------
% subHeading 

% 5 arguments:  position, date, company, location, description
\newcommand{\subHeading}[5]{
  % create new item in the list with reduced vertical space
  \vspace{-2pt}\item
  % if argument 5 is empty, print only 2 rows
  \ifthenelse{\isempty{#5}}
  {
    % table with 2 rows
    % spands 97% of the text width
    % [t] aligns content to the top
    % l@{\extracolsep{\fill}}r creates 2 columns with equal width
    % the first column is left aligned (l)
    % the second column is right aligned (r)
    % @{\extracolsep{\fill}} insert space that automatically streches to fill the width of the column
    \begin{tabular*}{0.97\textwidth}[t]{l@{\extracolsep{\fill}}r}
      % #1 is bold and #2 is normal
  		\textbf{#1} & #2 \\
      % #3 and #4 are small and italic
  		\textit{\small#3} & \textit{\small #4} \\
  	\end{tabular*}
    % reduce vertical space after the table
    \vspace{-7pt}
  }
  % else print 3 rows
  {
  	\begin{tabular*}{0.97\textwidth}[t]{l@{\extracolsep{\fill}}r}
  		\textbf{#1} & #2 \\
  		\textit{\small#3} & \textit{\small #4} \\
      % if #5 is not empty, print it in a new row
      % new row that spans the two columns
  		\multicolumn{2}{p{0.97\textwidth}}
        % description in small font size
        {\small #5}\\
  	\end{tabular*}\vspace{-7pt}
  }
}
% -------------------------
% subHeadingItemList

\newcommand{\subHeadingItemListStart}{\begin{itemize}}
\newcommand{\subHeadingItemListEnd}{\end{itemize}\vspace{-5pt}}

% -------------------------
% subHeadingItem

% new item with 1 argument: text
% small font size
% -3pt vertical space after the item
\newcommand{\subHeadingItem}[1]{
  % create new item in the list
  \item
  \small{{#1 \vspace{-3pt}}}
}
% this line of code is redefining the second-level itemize label 
% to be a smaller, vertically centered bullet point.
\renewcommand\labelitemii{$\vcenter{\hbox{\tiny$\bullet$}}$}

% -------------------------
% resumeSubItem 
% ????

\newcommand{\resumeSubItem}[1]{
  \subHeadingItem{#1}\vspace{-4pt}
}

% -------------------------
% subSubHeading

% This subHeading is for experiences that involve multiple positions
% 2 arguments:  position, date
\newcommand{\subSubHeading}[2]{
  % create new item in the list
  \item
    % new table with 2 columns like in subHeading
    \begin{tabular*}{0.97\textwidth}{l@{\extracolsep{\fill}}r}
      % #1 is bold and #2 is italic and small
      \textit{\small#1} & \textit{\small #2} \\
    \end{tabular*}\vspace{-7pt}
}

% -------------------------
% projectHeading

% 2 arguments
\newcommand{\projectHeading}[2]{
  % create new item in the list
    \item
    % new table with 2 columns like in subHeading
    \begin{tabular*}{0.97\textwidth}{l@{\extracolsep{\fill}}r}
      \small#1 & #2 \\
    \end{tabular*}\vspace{-7pt}
}